\chapter{External Interface}

The external interface of the TCU allows remote PEs to manipulate its state. First of all, the TCU
can be made privileged or unprivileged through the \texttt{FEATURES} register, which has the
following format:

\setlength{\regWidth}{.95\textwidth}
\begin{register}{H}{\texttt{FEATURES}}{0xF000\_0000}
  \regfield[gray!20]{reserved}{63}{1}{{uninitialized}}%
  \regfield{kernel}{1}{0}{1}%
  \reglabel{Reset}\regnewline%
  \begin{regdesc}\begin{reglist}
    \item[kernel] whether the TCU belongs to a kernel PE
  \end{reglist}\end{regdesc}
\end{register}
\setlength{\regWidth}{\textwidth}

\noindent In other words, the \texttt{FEATURES} register currently only defines whether the TCU
belongs to a kernel PE. At reset, \texttt{FEATURES.kernel} is set to 1, so that all PEs are kernel
PEs. The software starting on one PE can afterwards downgrade the other PEs to user PEs.

If \texttt{FEATURES.kernel} is 1, the CU can write to endpoint registers. This can be used to create
a communication channel to the TCU's MMIO region in another PE, which allows to establish other
communication channels by writing to the endpoint registers within the MMIO region.

Additionally, the TCU supports \emph{external commands}, which are triggered by writing to the
\texttt{EXT\_CMD} register and feedback is given via this register as well. The \texttt{EXT\_CMD}
register has the following format:

\begin{register}{H}{EXT\_CMD}{\texttt{0xF000\_0008}}
  \regfieldb{arg}{55}{9}%
  \regfieldb{err}{5}{4}%
  \regfieldb{op}{4}{0}\regnewline%
  \begin{regdesc}\begin{reglist}
    \item[arg] an argument for the operation
    \item[err] the result of the operation (output field)
    \item[op] the operation to execute
  \end{reglist}\end{regdesc}
\end{register}

\section{Command List}

The TCU supports the following external commands with the opcodes in parentheses. Other opcodes lead
to an \texttt{UNKNOWN\_CMD} error.

\begin{itemize}
  \item \texttt{IDLE} (0): don't do anything,
  \item \texttt{INV\_EP} (1): invalidate an endpoint.
\end{itemize}

\section{Pseudo Code Building Blocks}
\label{sec:extcmdspseudo}

The following sections use pseudo code to describe the behavior of the external TCU commands, based
on the following building blocks:

\begin{itemize}
  \item \texttt{ext\_stop(error)}:\\
  stop the execution of the external command by setting the opcode in \texttt{EXT\_CMD.op} to 0 and
  \texttt{EXT\_CMD.err} to the given error.
\end{itemize}

\section{Command Description}

\subsection{\texttt{INV\_EP}}

\begin{algorithm}[H]
    $epid \gets EXT\_CMD.arg\ \&\ 0xFFFF$\;
    $force \gets EXT\_CMD.arg >> 16$\;
    \BlankLine
    $EXT\_CMD.arg \gets 0$\;
    \BlankLine
    $ep \gets$ read\_ep(epid)\;
    \uIf{force == 0 and ep.type == SEND}{
      \uIf{ep.cur\_crd != ep.max\_crd}{ext\_stop(NO\_CREDITS)}
    }
    \uIf{force == 0 and ep.type == RECEIVE}{
      $EXT\_CMD.arg \gets ep.unread$\;
    }
    \BlankLine
    $ep \gets \{~type \gets INVALID~\}$\;
    $write\_ep(epid, ep)$\;
    \BlankLine
    $EXT\_CMD.err \gets NONE$\;
    $EXT\_CMD.op \gets 0$\;
    \caption{The TCU's \texttt{INV\_EP} command.}
\end{algorithm}
