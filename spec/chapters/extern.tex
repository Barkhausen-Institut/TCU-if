\chapter{External Interface}

The external interface of the TCU allows remote tiles to retrieve meta information about the tile
and TCU and manipulate its state.

\section{Meta Information}

The first part of the external interface are meta information about the TCU or the tile it belongs
to that can be retrieved and partially also be changed. The available TCU features can be configured
through the \texttt{FEATURES} register, which has the following format:

\setlength{\regWidth}{.95\textwidth}
\begin{register}{H}{\texttt{FEATURES}}{0xF000\_0000}
  \regfield{vpatch}{8}{56}{?}%
  \regfield{vminor}{8}{48}{?}%
  \regfield{vmajor}{16}{32}{?}%
  \regfield[gray!20]{reserved}{29}{3}{{uninitialized}}%
  \regfield{ctxsw}{1}{2}{0}%
  \regfield{vm}{1}{1}{0}%
  \regfield{kernel}{1}{0}{1}%
  \reglabel{Reset}\regnewline%
  \begin{regdesc}\begin{reglist}
    \item[vmajor] the TCU major version number
    \item[vminor] the TCU minor version number
    \item[vpatch] the TCU patch version number
    \item[ctxsw] whether the context-switching extension is available
    \item[vm] whether the virtual-memory extension is available
    \item[kernel] whether the TCU belongs to a kernel tile
  \end{reglist}\end{regdesc}
\end{register}
\setlength{\regWidth}{\textwidth}

\noindent At reset, \texttt{FEATURES.kernel} is set to 1, so that all tiles are kernel tiles. The
software starting on one tile can afterwards downgrade the other tiles to user tiles. If
\texttt{FEATURES.kernel} is 1, the CU can write to endpoint registers. This can be used to create a
communication channel to the TCU's MMIO region in another tile, which allows to establish other
communication channels by writing to the endpoint registers within the MMIO region.

The bits \texttt{FEATURES.ctxsw} and \texttt{FEATURES.vm} control whether the context-switching and
virtual-memory extension is enabled, respectively. If the former is disabled, foreign message
receptions do not raise a CU request. If the latter is disabled, no address translation is
performed and thus no translation CU request is raised. The behavior of privileged commands is
undefined if the associated extension is disabled.

The TCU uses semantic versioning\footnote{\url{https://semver.org}} to keep track of changes. Thus,
incompatible changes result in a new major version, compatible changes result in a new minor
version, and bugfixes result in a new patch version. All three version numbers are implementation
defined and can be used to verify whether the software is compatible to the TCU implementation at
hand. The version cannot be changed at runtime. The major and minor version of the implementation
should correspond to the major and minor version of the specification it implements, which is shown
on the title page of this document.

Besides \texttt{FEATURES}, the TCU provides the register \texttt{TILE\_DESC} that contains a
description of the tile. The description is set by the hardware and cannot be changed.

\begin{register}{H}{TILE\_DESC}{\texttt{0xF000\_0008}}
  \regfieldb{memory}{36}{28}%
  \regfieldb{attr}{17}{11}%
  \regfieldb{isa}{5}{6}%
  \regfieldb{type}{6}{0}\regnewline%
  \begin{regdesc}\begin{reglist}
    \item[memory] the internal memory size in 4~KiB pages
    \item[attr] additional attributes with the following bits:
    \begin{itemize}
      \item \texttt{BOOM} (1): the tile contains a BOOM core
      \item \texttt{ROCKET} (2): the tile contains a Rocket core
      \item \texttt{NIC} (4): a network interface card attached to the core
      \item \texttt{SERIAL} (8): a serial interface attached to the core
      \item \texttt{IMEM} (16): the tile has an internal memory
    \end{itemize}
    \item[isa] the instruction set architecture:
    \begin{itemize}
      \item \texttt{NONE} (0): dummy ISA for memory tiles
      \item \texttt{RISCV} (1)
    \end{itemize}
    \item[type] the type of tile:
    \begin{itemize}
      \item \texttt{COMP} (0): contains a processor for computation
      \item \texttt{MEM} (1): contains memory
    \end{itemize}
  \end{reglist}\end{regdesc}
\end{register}

\noindent Note that other values and bits are currently not defined. The memory size is 0 if the
attribute \texttt{IMEM} is not set and the size of the internal memory in 4~KiB pages otherwise.

\section{External Commands}

The TCU supports \emph{external commands}, which are triggered by writing to the \texttt{EXT\_CMD}
register and feedback is given via this register as well. The \texttt{EXT\_CMD} register has the
following format:

\begin{register}{H}{EXT\_CMD}{\texttt{0xF000\_0010}}
  \regfieldb{arg}{55}{9}%
  \regfieldb{err}{5}{4}%
  \regfieldb{op}{4}{0}\regnewline%
  \begin{regdesc}\begin{reglist}
    \item[arg] an argument for the operation
    \item[err] the result of the operation (output field)
    \item[op] the operation to execute
  \end{reglist}\end{regdesc}
\end{register}

The TCU supports the following external commands with the opcodes in parentheses. Other opcodes lead
to an \texttt{UNKNOWN\_CMD} error.

\begin{itemize}
  \item \texttt{IDLE} (0): don't do anything,
  \item \texttt{INV\_EP} (1): invalidate an endpoint.
\end{itemize}

\section{Pseudo Code Building Blocks}
\label{sec:extcmdspseudo}

The following sections use pseudo code to describe the behavior of the external TCU commands, based
on the following building blocks:

\begin{itemize}
  \item \texttt{ext\_stop(error)}:\\
  stop the execution of the external command by setting the opcode in \texttt{EXT\_CMD.op} to 0 and
  \texttt{EXT\_CMD.err} to the given error.
\end{itemize}

\section{Command Description}

\subsection{\texttt{INV\_EP}}

\begin{algorithm}[H]
    $epid \gets EXT\_CMD.arg\ \&\ 0xFFFF$\;
    $force \gets EXT\_CMD.arg >> 16$\;
    \BlankLine
    $EXT\_CMD.arg \gets 0$\;
    \BlankLine
    $ep \gets$ read\_ep(epid)\;
    \uIf{force == 0 and ep.type == SEND}{
      \uIf{ep.cur\_crd != ep.max\_crd}{ext\_stop(NO\_CREDITS)}
    }
    \uIf{force == 0 and ep.type == RECEIVE}{
      $EXT\_CMD.arg \gets ep.unread$\;
    }
    \BlankLine
    $ep \gets \{~type \gets INVALID~\}$\;
    $write\_ep(epid, ep)$\;
    \BlankLine
    $EXT\_CMD.err \gets NONE$\;
    $EXT\_CMD.op \gets 0$\;
    \caption{The TCU's \texttt{INV\_EP} command.}
\end{algorithm}
