\chapter{Privileged Interface}

The privileged interface of the TCU adds support for virtual memory and PE sharing to the TCU and is
thus only required if either of these features are desired. Thus, all non-highlighted functionality
is required if either extension is used, whereas the highlighted functionality is only required for
one specific extension.

The interface consists of multiple privileged registers, privileged commands, and \emph{core
requests}, which are raised by the TCU and answered by the privileged software. To support the
privileged software in the management of multiple VPEs or virtual memory, privileged commands are
used, which can be performed by writing to \texttt{PRIV\_CMD}. The register is defined as follows:

\begin{register}{H}{PRIV\_CMD}{\texttt{0xF000\_2010}}
  \regfieldb{arg}{60}{4}%
  \regfieldb{op}{4}{0}\regnewline%
  \begin{regdesc}\begin{reglist}
    \item[arg] the argument for the privileged command
    \item[op] the opcode of the privileged command
  \end{reglist}\end{regdesc}
\end{register}

\extstart{pemux}
\noindent The TCU stores the id of the current VPE including the number of unread messages in all
its receive EPs in the \texttt{CUR\_VPE} register:

\setlength{\regWidth}{.95\textwidth}
\begin{register}{H}{CUR\_VPE}{\texttt{0xF000\_2020}}
  \regfield[gray!20]{reserved}{32}{32}{{uninitialized}}%
  \regfield{msgs}{16}{16}{0}%
  \regfield{id}{16}{0}{1111111111111111}%
  \reglabel{Reset}\regnewline%
  \begin{regdesc}\begin{reglist}
    \item[msgs] the sum of all unread messages in the receive EPs for this VPE
    \item[id] the VPE id
  \end{reglist}\end{regdesc}
\end{register}
\setlength{\regWidth}{\textwidth}
\extend{}

\section{Command List}

The following privileged commands are supported with the opcodes in parentheses:

\begin{itemize}
  \item \texttt{IDLE} (0): \extstart{vmpex} don't do anything, \extend{}
  \item \texttt{INV\_PAGE} (1): invalidate a page in the TLB, \extstart{vm}
  \item \texttt{INV\_TLB} (2): invalidate the complete TLB,
  \item \texttt{INS\_TLB} (3): insert an entry into the TLB, \extend{}
  \item \texttt{XCHG\_VPE} (4): change the VPE, \extstart{pemux}
  \item \texttt{SET\_TIMER} (5): start/stop the timer,
  \item \texttt{ABORT\_CMD} (6): abort the unprivileged command. \extend{}
\end{itemize}

\section{Pseudo Code Building Blocks}

The following sections use pseudo code to describe the behavior of the privileged commands, based on
several building blocks:

\begin{itemize}
  \item \texttt{insert\_tlb\_entry(asid, virt, phys, flags)}: \extstart{vm}\\
  inserts an entry from \texttt{virt} to \texttt{phys} into the TLB for address space \texttt{asid}
  \item \texttt{remove\_tlb\_entry(asid, virt)}:\\
  removes the entry with \texttt{virt} in address space \texttt{asid}
  \item \texttt{clear\_tlb()}:\\
  removes all entries from the TLB \extend{}
  \item \texttt{set\_timer(timeout)}: \extstart{pemux}\\
  starts/restarts the timer to fire an interrupt after \texttt{timeout} nanoseconds
  \item \texttt{unset\_timer()}:\\
  stops the timer
  \item \texttt{waiting\_for\_resp() -> bool}:\\
  returns whether the TCU is currently waiting for a remote TCU's response (to \texttt{SEND},
  \texttt{REPLY}, \texttt{READ}, or \texttt{WRITE})
  \item \texttt{wait\_for\_resp()}:\\
  waits until a remote TCU responds (to \texttt{SEND}, \texttt{REPLY}, \texttt{READ}, or
  \texttt{WRITE})
  \item \texttt{abort\_remote\_xfer()}:\\
  lets the current remote transfer fail, so that \texttt{read\_remote\_phys()} or \texttt{write\_remote\_phys()}
  return the error ABORT (see \rref{sec:unprivcmdspseudo})
  \item \texttt{abort\_local\_xfer() -> bool}:\\
  checks if there is a local transfer (for the current unprivileged command) and if so, aborts it,
  so that \texttt{read\_local()} or \texttt{write\_local()} return the error ABORT (see
  \rref{sec:unprivcmdspseudo}). Returns true if it was aborted. \extend{}
\end{itemize}

\section{Command Description}

\extstart{vm}
\subsection{\texttt{INV\_PAGE}}

\begin{algorithm}[H]
    $asid \gets PRIV\_CMD.arg0 >> 32$\;
    $virt \gets PRIV\_CMD.arg0\ \&\ 0xFFFF\_FFFF$\;
    \BlankLine
    $remove\_tlb\_entry(asid, virt)$\;
    $PRIV\_CMD.op \gets 0$\;
    \caption{The TCU's \texttt{INV\_PAGE} command.}
\end{algorithm}

\subsection{\texttt{INV\_TLB}}

\begin{algorithm}[H]
    $clear\_tlb()$\;
    $PRIV\_CMD.op \gets 0$\;
    \caption{The TCU's \texttt{INV\_TLB} command.}
\end{algorithm}

\subsection{\texttt{INS\_TLB}}

\begin{algorithm}[H]
    $asid \gets PRIV\_CMD.arg0 >> 32$\;
    $virt \gets PRIV\_CMD.arg0\ \&\ 0xFFFF\_F000$\;
    $flags \gets PRIV\_CMD.arg0\ \&\ 0x1F$\;
    $phys \gets PRIV\_CMD\_ARG1$\;
    \BlankLine
    $insert\_tlb\_entry(asid, virt, phys, flags)$\;
    $PRIV\_CMD.op \gets 0$\;
    \caption{The TCU's \texttt{INS\_TLB} command.}
\end{algorithm}
\extend{}

\extstart{pemux}
\subsection{\texttt{XCHG\_VPE}}

\begin{algorithm}[H]
    $old \gets CUR\_VPE$\;
    $OLD\_VPE \gets old$\;
    $CUR\_VPE \gets PRIV\_CMD.arg0\ \&\ 0xFFFF\_FFFF$\;
    \BlankLine
    $PRIV\_CMD.op \gets 0$\;
    \caption{The TCU's \texttt{XCHG\_VPE} command.}
\end{algorithm}

\subsection{\texttt{SET\_TIMER}}

\begin{algorithm}[H]
    $nanos \gets PRIV\_CMD.arg0$\;
    \uIf{nanos == 0}{
      $unset\_timer()$\;
    }
    \uElse{
      $set\_timer(nanos)$\;
    }
    \BlankLine
    $PRIV\_CMD.op \gets 0$\;
    \caption{The TCU's \texttt{SET\_TIMER} command.}
\end{algorithm}

\subsection{\texttt{ABORT\_CMD}}

\begin{algorithm}[H]
    \uIf{COMMAND.op != 0}{
      \uIf{waiting\_for\_resp()}{
        \uIf{COMMAND.op == READ or COMMAND.op == WRITE}{
          $PRIV\_CMD.arg0 \gets 1$\;
          $abort\_remote\_xfer()$\;
        }
        $wait\_for\_resp()$\;
      }
      \uElseIf{abort\_local\_xfer()}{
        $PRIV\_CMD.arg0 \gets 1$\;
      }
    }
    \BlankLine
    $PRIV\_CMD.op \gets 0$\;
    \caption{The TCU's \texttt{ABORT\_CMD} command.}
\end{algorithm}
\extend{}

\section{Core Requests}

With the privileged interface to support virtual memory and PE sharing, the TCU needs assistance of
the CU in some cases. Since we assume in this case that the CU is a general-purpose core, these
requests for assistance are called \emph{core requests}. Core requests are needed, for example, if a
message for another VPE than the currently running VPE (\texttt{CUR\_VPE}) is received.

Since multiple core requests might occur simultaneously, the TCU keeps a queue of pending core
requests and handles them in FIFO order. To handle the first in the queue, the TCU writes to
\texttt{CORE\_REQ} and injects an interrupt to inform the core about the request. After the handling
of the core request, the core is expected to write to \texttt{CORE\_REQ} again to signal the
completion of the request. Upon receiving this signal, the TCU issues the next core request, if
there is any. New core requests are appended to the queue via \texttt{queue\_foreign\_msg\_req()}
(see \rref{sec:unprivcmdspseudo}) and \texttt{queue\_xlate\_req()} (see \rref{sec:xlates}).

\begin{register}{H}{CORE\_REQ}{\texttt{0xF000\_2000}}
  \regfieldb[gray!60]{}{63}{1}%
  \regfieldb{type}{1}{0}\regnewline%
  \begin{regdesc}\begin{reglist}
    \item[type] the type of CU request (0 = translation, 1 = foreign message)
  \end{reglist}\end{regdesc}
\end{register}

\noindent As the \texttt{CORE\_REQ} register is used for different kinds of requests and responses,
most of the bits depend on the type of request/response, defined in the following.

\extstart{vm}
\begin{register}{H}{CORE\_REQ for translation requests}{\texttt{0xF000\_2000}}
  \regfieldb{vpe}{16}{48}%
  \regfieldb{virt}{36}{12}%
  \regfieldb[gray!20]{reserved}{8}{4}%
  \regfieldb{access}{2}{2}%
  \regfieldb{pf}{1}{1}%
  \regfieldb{type}{1}{0}\regnewline%
  \begin{regdesc}\begin{reglist}
    \item[pf] whether the translation may page fault or should fail in this case
    \item[access] the requested access permissions (see \rref{sec:tlb})
    \item[virt] the virtual address to translate (shifted right by 12 bits)
    \item[vpe] the id of the VPE for whom to perform the translation
  \end{reglist}\end{regdesc}
\end{register}

\begin{register}{H}{CORE\_REQ for translation responses}{\texttt{0xF000\_2000}}
  \regfieldb{phys}{52}{12}%
  \regfieldb[gray!20]{reserved}{7}{5}%
  \regfieldb{flags}{5}{0}\regnewline%
  \begin{regdesc}\begin{reglist}
    \item[flags] the flags for the TLB entry (see \rref{sec:tlb})
    \item[phys] the physical address for the TLB entry
  \end{reglist}\end{regdesc}
\end{register}
\extend{}

\extstart{pemux}
\begin{register}{H}{CORE\_REQ for foreign message requests}{\texttt{0xF000\_2000}}
  \regfieldb{vpe}{16}{48}%
  \regfieldb[gray!20]{reserved}{31}{17}%
  \regfieldb{ep}{16}{1}%
  \regfieldb{type}{1}{0}\regnewline%
  \begin{regdesc}\begin{reglist}
    \item[ep] the receive EP which received a message
    \item[vpe] the VPE that received the message
  \end{reglist}\end{regdesc}
\end{register}

\begin{register}{H}{CORE\_REQ for foreign message responses}{\texttt{0xF000\_2000}}
  \regfieldb[gray!20]{reserved}{64}{0}\regnewline%
\end{register}
\extend{}
